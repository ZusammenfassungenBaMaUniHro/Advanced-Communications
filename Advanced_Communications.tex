 \documentclass{article} %A4
\usepackage[a4paper,left=1.9cm, right=2.1cm,top = 1.2cm,bottom=2.3cm]{geometry}
\usepackage[utf8]{inputenc}%Umlaute
\usepackage[ngerman]{babel} %Texttrennung
\usepackage{graphicx}	%Grafiken
\usepackage{amssymb}
\usepackage{amsmath}
\usepackage{url}
\usepackage{listings}
\usepackage{color}
\usepackage{hyperref}
\usepackage{framed}
\usepackage{algpseudocode}
\usepackage{tikz}
\usepackage{enumitem}

\usepackage[labelformat=empty]{caption}
\title{Zusammenfassung - Kryptographie}
\author{
	Marc Meier
}

\begin{document}
\maketitle
\begin{framed}Korrektheit und Vollständigkeit der Informationen sind nicht gewährleistet.
Macht euch eigene Notizen oder ergänzt/korrigiert meine Ausführungen!
\end{framed}
\setcounter{tocdepth}{1}
\tableofcontents

\section{Grundlagen}

% 010, 020
\subsection{Grundprinzipien und Entwicklung des Internets}

\subsection{Dezentrale Verwaltung des Internets}

\subsection{Standards}

\subsection{Netze, Autonome Systeme und Schichten}

\section{Protokolle}

\subsection{OSI-7-Schichten-Modell}

\subsection{Zustandslose und zustandsbehaftete Protokolle}

\subsection{Ethernet}


\subsection{Switching}

\subsection{Asynchronous Transfer Mode}

\subsection{Internet Protocol}

\subsubsection{IPv4}

\subsubsection{IPv6}

\subsection{User Datagram Protocol}

\subsection{Transmission Control Protocol}

\section{Adressierung}

% 030

\section{ARP, RARP}

% 040

\section{DNS und WHOIS}

% 050

\section{Migration von IPv4 nach IPv6}

% 060

\section{Timeouts, ACK, Bestätigungen}

% 070

\section{Routingkonzepte}

% 080, 090, 095, 100, 110, 120, 130

\section{Quality of Service}

% 140

\section{Multicasts}

% 150, 180

\section{Zeitsynchronisation}

% 160, 190

\section{Internet Control Message Protocol}

% 170

\section{Voice over IP}

% 200, 210

\section{World Wide Web und HTTP}

% 220, 230

\section{Peer-to-Peer}

% 240, 250

\section{E-Mail}

% 260

\section{Autokonfiguration}

% 280, 290

\section{Dateien und Drucken}

% 300, 310

\section{Telnet, SSH und rlogin}

% 320

\section{Extensible Messaging and Presence Protocol (XMPP)}

% 330

\section{LDAP}

% 340

\section{Authentication Protocols}

% 350

\section{Simple Network Management Protocol}

% 360 

\section{Mac-Sublayer}

% 365

\section{Mobile Netzwerke}

% 370, 380, 390, 400

\section{HTTP2 und SCTP}

% 410


\newpage
\nocite{*}
\addcontentsline{toc}{section}{Literatur}
\bibliographystyle{plain}

\bibliography{literatur}
\end{document}