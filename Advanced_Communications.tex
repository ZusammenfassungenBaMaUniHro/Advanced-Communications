 \documentclass{article} %A4
\usepackage[a4paper,left=1.9cm, right=2.1cm,top = 1.2cm,bottom=2.3cm]{geometry}
\usepackage[utf8]{inputenc}%Umlaute
\usepackage[ngerman]{babel} %Texttrennung
\usepackage{graphicx}	%Grafiken
\usepackage{amssymb}
\usepackage{amsmath}
\usepackage{url}
\usepackage{listings}
\usepackage{color}
\usepackage{hyperref}
\usepackage{framed}
\usepackage{algpseudocode}
\usepackage{tikz}
\usepackage{enumitem}
\usepackage[xindy]{glossaries}
\makeglossaries
\usepackage[labelformat=empty]{caption}
\title{Zusammenfassung - Kryptographie}
\author{
	Marc Meier
}




\begin{document}
\maketitle
\begin{framed}Korrektheit und Vollständigkeit der Informationen sind nicht gewährleistet.
Macht euch eigene Notizen oder ergänzt/korrigiert meine Ausführungen!
\end{framed}
\setcounter{tocdepth}{1}
\tableofcontents

\section{Grundlagen}

% 010, 020
\subsection{Grundprinzipien und Entwicklung des Internets}
Das Internet entwickelte sich ab den 1960er Jahren.
Es ging aus dem am Ende des Jahrzehnts entstandenen, vornehmlich militärisch und akademisch geprägten ARPA-Net hervor.
Heutzutage wird es international kommerziell, industriell und auch akademisch (Katzenbilder) genutzt.
Bei seiner Entstehung war vor allem eine dezentrale Struktur ohne zentrale Verwaltung von Interesse.
Grund hierfür war die Angst des amerikanischen Department of Defense, dass eine atomarer Angriff zentrale Kommunikationspunkte außer Kraft setzen könnte.
Die Kommunikation findet über hochgradig vernetzte Knoten mithilfe von Paketen statt.

Literatur: \cite{abbate2000inventing, baran1964distributed}
\subsection{Packet Switching}
Paketvermittelte Übertragung bedeutet die Abkehr von der leitungsbasierten Vermittlung.
Dabei werden längere Nachrichten in Datenpakete aufgeteilt und voneinander unabhängig versendet.
Dies ermöglicht eine faire Verteilung der Leistungskapazität und redundante Wege bei einem Ausfall von Knoten oder Verbindungen.
Im Gegenzug können konstante Bandbreiten nicht ohne Weiteres (Abschnitt \ref{sec:qos}) gewährleistet werden, ebenso ergeben sich unterschiedliche Laufzeiten von Paketen.

\subsection{Dezentrale Verwaltung des Internets}

\subsubsection{Prinzipien}

\begin{itemize}
\item Keine zentrale Verwaltung oder Behörde (trotz Einflussnahme)
\item Demokratisches Zusammenwirken der Beteiligten / Wahlen
\item Selbstorganisation
\item Standards dort, wo sie erforderlich sind
\item Dynamisch, offen für Neuigkeiten	
\end{itemize}

\subsubsection{Organisationen}

\begin{description}
\item [ICANN:] Vergibt IP-Adressen und betreibt die DNS-Rootserver.
\item[IETF:] Standardisierung von Protokollen in RFCs \cite{rfc3233}
\item[RIPE:] Administration und technische Koordination
\item[RIPE NCC:] Adressvergabe in Europa und Zentralasien, Verwaltung der WHOIS-Datenbank.
\item[DENIC eG:] Domain-Verwaltung für die Zone .de
\end{description}

\subsection{Standards}
Standards ermöglichen die Kooperation im Netzwerk, nur durch sie können Geräte verschiedener Hersteller miteinander kommunizieren.
Sie können textuell, mithilfe einer Referenzimplementierung oder anhand von Automaten (meist für zustandsbehaftete Protokolle) festgelegt werden.

\begin{description}
\item[Protokoll] Standardisierte Regeln (Vorschriften) und Vereinbarungen zu
Form, Ablauf, Steuerung und Sicherung (Fehler) der
Datenübertragung in und zwischen Rechnernetzen, zwischen
Einzel-Rechnern und zwischen Rechnern und
Peripheriegeräten.
\item[Standard] Ein Standard wird von den verschiedensten internationalen und
nationalen Organisationen sowie von großen Firmen erstellt.
Ein Standard wird als verbindliche oder unverbindliche
(empfohlene) Festlegungen schriftlich niedergelegt.
\end{description}
\subsection{Netze, Autonome Systeme und Schichten}

\section{Protokolle}

\subsection{OSI-7-Schichten-Modell}

\subsection{Zustandslose und zustandsbehaftete Protokolle}

\subsection{Ethernet}


\subsection{Switching}

\subsection{Asynchronous Transfer Mode}

\subsection{Internet Protocol}

\subsubsection{IPv4}

\subsubsection{IPv6}

\subsection{User Datagram Protocol}

\subsection{Transmission Control Protocol}

\section{Adressierung}

% 030

\section{ARP, RARP}

% 040

\section{DNS und WHOIS}

% 050

\section{Migration von IPv4 nach IPv6}

% 060

\section{Timeouts, ACK, Bestätigungen}

% 070

\section{Routingkonzepte}

% 080, 090, 095, 100, 110, 120, 130

\section{Quality of Service}
\label{sec:qos}
% 140

\section{Multicasts}

% 150, 180

\section{Zeitsynchronisation}

% 160, 190

\section{Internet Control Message Protocol}

% 170

\section{Voice over IP}

% 200, 210

\section{World Wide Web und HTTP}

% 220, 230

\section{Peer-to-Peer}

% 240, 250

\section{E-Mail}

% 260

\section{Autokonfiguration}

% 280, 290

\section{Dateien und Drucken}

% 300, 310

\section{Telnet, SSH und rlogin}

% 320

\section{Extensible Messaging and Presence Protocol (XMPP)}

% 330

\section{LDAP}

% 340

\section{Authentication Protocols}

% 350

\section{Simple Network Management Protocol}

% 360 

\section{Mac-Sublayer}

% 365

\section{Mobile Netzwerke}

% 370, 380, 390, 400

\section{HTTP2 und SCTP}

% 410


\newpage
\addcontentsline{toc}{section}{Literatur}
\bibliographystyle{plain}
\bibliography{literatur}


\newglossaryentry{arpa}{name=ARPA,description={Advanced Research Projects Agency}}
\newglossaryentry{icann}{name=ICANN,description={Internet Corporation for Assigned Names and Numbers}}
\newglossaryentry{ietf}{name=IETF,description={Internet Engineering Task Force}}
\newglossaryentry{ripe}{name=RIPE,description={Réseaux IP Européens}}
\newglossaryentry{ripencc}{name=RIPE NCC,description={Réseaux IP Européens Network Coordination Centre}}
\newglossaryentry{denic}{name=DENIC,description={Deutsches Network Information Center}}
\newpage
\addcontentsline{toc}{section}{Glossar}

\printglossary[title=List of Terms,toctitle=Terms and abbreviations]

\end{document}